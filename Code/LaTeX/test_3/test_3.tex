\documentclass[12pt, a4paper, oneside]{ctexart}
\usepackage{amsmath, amsthm, amssymb, appendix, bm, graphicx, hyperref, mathrsfs}
\usepackage{booktabs}
\usepackage{cite}
\usepackage[numbers,sort&compress]{natbib}
\newcommand{\upcite}[1]{\textsuperscript{\textsuperscript{\cite{#1}}}}
\title{\textbf{论文标题}}
\author{Dylaaan}
\date{\today}
\linespread{1.5}
\newtheorem{theorem}{定理}[section]
\newtheorem{definition}[theorem]{定义}
\newtheorem{lemma}[theorem]{引理}
\newtheorem{corollary}[theorem]{推论}
\newtheorem{example}[theorem]{例}
\newtheorem{proposition}[theorem]{命题}
\renewcommand{\abstractname}{\Large\textbf{摘要}}

\begin{document}

\maketitle

\setcounter{page}{0}
\maketitle
\thispagestyle{empty}

\begin{abstract}
    这里是摘要.
    \par\textbf{关键词:}这里是关键词; 这里是关键词.
\end{abstract}

\newpage
\pagenumbering{Roman}
\setcounter{page}{1}
\tableofcontents
\newpage
\setcounter{page}{1}
\pagenumbering{arabic}

\section{一级标题}

\subsection{二级标题}

这里是正文\upcite{wk}.你好你好你说地方噶是的发生的\upcite{Mishra2023}
\newpage
Reference
\bibliographystyle{unsrt}      %参考文献样式
\bibliography{C:/Users/IK/Desktop/testbib}

\begin{appendices}
    \renewcommand{\thesection}{\Alph{section}}
    \section{附录标题}
    这里是附录.
\end{appendices}

\end{document}