\documentclass{mcmthesis}
\usepackage{xcolor} % 加入xcolor宏包,改变字体颜色
\usepackage{indentfirst} %首行缩进
\setlength{\parindent}{2em} %2em代表首行缩进两个字符
\mcmsetup{
	CTeX = false,
	tcn =\textcolor{red}{2411262},
	problem =\textcolor{red}{A},
	sheet = true, titleinsheet = true, keywordsinsheet = true,
	titlepage = false, abstract = true}
\usepackage{newtxtext}%\usepackage{palatino}
\usepackage{lipsum}
\usepackage{float}
\usepackage{indentfirst}
\usepackage{multirow}
\setlength{\headheight}{14.5pt} % 增加头部高度
\addtolength{\topmargin}{-2.5pt} % 调整顶部边距以补偿
\title{\textbf{Rapid Bushfire Response for Emergency Response}}
\begin{document}
\begin{abstract}
	\qquad How to respond and deal with fires in time when they occur is a question worth
	thinking about. This paper provides a fire response plan for Victoria through the ratio-
	nal deployment of EOC, drones and forward teams.

	In Task 1, the paper establishes the area safety evaluation model with fire frequency,
	size, andrecentfiresituationasindicators toclassify thedanger levels of different areas
	in Victoria. Then, we confirm that drones should provide different services for high-
	risk and dangerous areas. In order to increase the service capacity, we optimize both
	capacity and cost. For capability, we make the average response time of SSA drones
	to high-risk areas as short as possible, and reserve as many SSA drones to dangerous
	areas as possible. For cost, we quantify the demand for SSA drones in terms of fire
	acreage, and take into account the rounds and the attrition rate of drones. We also
	consider the mix between SSA drones and radio repeater drones, and calculate the
	number of repeater drones using a greedy mix-based maximum number solving algo-
	rithm. The total cost is calculated and finally the quantity optimization model based on
	the maximum mix rate and minimum cost is obtained.\textbf{ 200 SSA drones and 32 radio
		repeater drones are needed, respectively.}
	\begin{keywords}
		safety factor \quad commensurable \quad signal-to-noise ratio
	\end{keywords}
\end{abstract}

\maketitle



\clearpage
\pagestyle{fancy}
% Uncomment the next line to generate a Table of Contents
%\tableofcontents
\newpage
\setcounter{page}{1}
\rhead{Page \thepage\ }

\newpage   %%目录页
\tableofcontents
\thispagestyle{empty}
\newpage

%%%%%%%%%%%%%%%%%%%%%%%%%%%%%%%%%%%%%%%%%%%%%%%%%%%%%%%%%%%%
\section{Introduction}
\subsection{Problem Background}
PROBLEM BACKGROUND
\subsection{Restatement of the Problem}
RESTATEMENT
\subsection{Our Work}
ONE FIGURE AND SOME DISCRIPTION.

%%%%%%%%%%%%%%%%%%%%%%%%%%%%%%%%%%%%%%%%%%%%%%%%%%%%%%%%%%%%
\section{Assumptions and Notations}
\subsection{Assumptions}
\begin{itemize}
	\item
	\item
	\item
	\item
\end{itemize}
\subsection{Notations}
NOTATIONS

%%%%%%%%%%%%%%%%%%%%%%%%%%%%%%%%%%%%%%%%%%%%%%%%%%%%%%%%%%%%
\section{Model I: XXXXXX}
PRE-DISCRIPTION OF THE MODEL. SOME FIGURES AS WELL.
\subsection{Factors/Development of the model}
BALABALA
\subsubsection{More Sub-details I}
BALABALA
\subsubsection{More Sub-details II}
BALABALA
\subsection{Factors/Development of the model}
BALABALA
\subsection{Factors/Development of the model}
BALABALA

%%%%%%%%%%%%%%%%%%%%%%%%%%%%%%%%%%%%%%%%%%%%%%%%%%%%%%%%%%%%
\section{Model II: XXXXXX}
PRE-DISCRIPTION OF THE MODEL. SOME FIGURES AS WELL.
\subsection{Factors/Development of the model}
BALABALA
\subsubsection{More Sub-details I}
BALABALA
\subsubsection{More Sub-details II}
BALABALA
\subsection{Factors/Development of the model}
BALABALA
\subsection{Factors/Development of the model}
BALABALA

%%%%%%%%%%%%%%%%%%%%%%%%%%%%%%%%%%%%%%%%%%%%%%%%%%%%%%%%%%%%
\section{Application of Models}
CHOOSE CERTAIN REGION TO UTILIZE THE MODELS ABOVE.

%%%%%%%%%%%%%%%%%%%%%%%%%%%%%%%%%%%%%%%%%%%%%%%%%%%%%%%%%%%%
\section{Sensitivity Analysis}
NECESSARY SENSITIVITY ANALYSIS HERE.

%%%%%%%%%%%%%%%%%%%%%%%%%%%%%%%%%%%%%%%%%%%%%%%%%%%%%%%%%%%%
\section{Model Evaluation and Further Discussion}
\subsection{Strengths}
\begin{itemize}
	\item
	\item
	\item
	\item
\end{itemize}
\subsection{Weakness}
\begin{itemize}
	\item
	\item
\end{itemize}
\subsection{Further Discussion}
\begin{itemize}
	\item
	\item
\end{itemize}

%%%%%%%%%%%%%%%%%%%%%%%%%%%%%%%%%%%%%%%%%%%%%%%%%%%%%%%%%%%%
\section{Newspaper Artical/Flyer/Magazine}
BALABALA

%%%%%%%%%%%%%%%%%%%%%%%%%%%%%%%%%%%%%%%%%%%%%%%%%%%%%%%%%%%%
\newpage
\newpage
\begin{thebibliography}{99}
	\bibitem{1}
	\bibitem{2}
	\bibitem{3}
	\bibitem{4}
	\bibitem{5}
	\bibitem{6}
	\bibitem{7}
	\bibitem{8}
	\bibitem{9}
	\bibitem{10}
	\bibitem{11}
	\bibitem{12}
	\bibitem{13}
\end{thebibliography}

%%%%%%%%%%%%%%%%%%%%%%%%%%%%%%%%%
\newpage
\newpage

\newpage
\newpage
\begin{appendices}
	\section{First appendix}
	\begin{lstlisting}
		CODE1
		2
		3
		4
		5
	\end{lstlisting}
\end{appendices}

\end{document}
\end